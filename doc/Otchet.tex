\documentclass[a4paper,14pt,oneside]{extreport}  %односторонняя печать
\usepackage[utf8]{inputenc}
\usepackage[russianb]{babel}
\usepackage{fontspec}
\usepackage {titlesec}
\usepackage{tocloft}
\setlength{\cftbeforetoctitleskip}{-1em}
\titleformat{\chapter}{\thispagestyle{plain}\hyphenpenalty=10000\normalfont\huge\bfseries}{
	\thechapter. }{0pt}{\Huge}
\makeatother
\makeatletter
\renewcommand{\@makeschapterhead}[1]{\vspace{0pt}{\parindent=0pt \raggedright\normalfont\Huge\bfseries #1\par\nopagebreak\vspace{20 pt}}}
\makeatother
\titlespacing{\chapter}{0pt}{-1cm}{1.5cm}
\setmainfont{Times  New Roman}
\usepackage{vmargin}
\usepackage{amsmath}
\setpapersize{A4}
\setmarginsrb{3cm}{2cm}{1cm}{2cm}{0pt}{0mm}{0pt}{13mm}
\usepackage{indentfirst}
\sloppy
\begin{document}
	\begin{titlepage}
		\begin{centering}
		{ МИНИСТЕРСТВА НАУКИ И ВЫСШЕГО ОБРАЗОВАНИЯ \\ РОССИЙСКОЙ ФЕДЕРАЦИИ\\Федеральное государственное автономное образовательное учреждение\\
			высшего образования «Самарский национальный исследовательский\\
			университет имени академика С.П. Королева»\\{(Самарский университет)}\\	}
      	\end{centering}
      \vfill
        	\begin{table}[h]
            \begin{center}
         	\begin{tabular}{|p{2,94cm}|p{12,76cm}|}
       		\hline Институт & информатики и кибернетики \\ \hline
       		Факультет & информатики  \\ \hline
       		Кафедра & геоинформатики и информационной безопасности \\ \hline
        	\end{tabular}
            \end{center}
            \end{table}
		\vfill
		\centerline{КУРСОВОЙ ПРОЕКТ ПО ДИСЦИПЛИНЕ} 
		\vskip6pt
		\centerline{«Системное программирование»}
		\vfill
        \centerline {ПОЯСНИТЕЛЬНАЯ ЗАПИСКА}
        \vfill
         \centerline {«ОТЛАДЧИК»}
         \vfill
          	\begin{table}[h]
          		\begin{center}
          			\begin{tabular}{|p{12,25cm}|p{3,5cm}|}
        			\hline  & 
        			\\ Студент\hrulefill  & Хвацкова А.А. \\
        			\ \ \ \ \ \ \ \ \ \ \ \ \ \ \ \ \ \ \ \ \ \ \ \ \ \ \ \ \ \ \ \ \ \ \ \ \ \ \ \ \ \ \ \ \ \ \ \ \ \ \ \ \ \ \ \ \ \ \ \ \ \ \ \ \itshape ({\small подпись}) &   \\  \hline  &
        			\\  Преподаватель\hrulefill  &  Борисов А.Н.\\ 
        			\ \ \ \ \ \ \ \ \ \ \ \ \ \ \ \ \ \ \ \ \ \ \ \ \ \ \ \ \ \ \ \ \ \ \ \ \ \ \ \ \ \ \ \ \ \ \ \ \ \ \ \ \ \ \ \ \ \ \ \ \ \ \ \ \itshape ({\small подпись}) & \\  \hline
        		\end{tabular}
        	\end{center}
        		\end{table}
        \vfill
        \vfill
        \vfill
        \vfill
        \vfill
        \vfill
        \vfill
        \vfill
        \vfill
        \vfill
        \vfill
        \vfill
        \centerline{Самара~--- 2022}
	\end{titlepage}

\begin{titlepage}
		\begin{centering}
		{ МИНИСТЕРСТВА НАУКИ И ВЫСШЕГО ОБРАЗОВАНИЯ \\ РОССИЙСКОЙ ФЕДЕРАЦИИ\\Федеральное государственное автономное образовательное учреждение\\
			высшего образования «Самарский национальный исследовательский\\
			университет имени академика С.П. Королева»\\{(Самарский университет)}\\	}
     	\end{centering}
    	\begin{table}[h]
		\begin{center}
			\begin{tabular}{|p{2,94cm}|p{12,76cm}|}
				\hline Институт & информатики и кибернетики \\ \hline
				Факультет & информатики  \\ \hline
				Кафедра & геоинформатики и информационной безопасности \\ \hline
			\end{tabular}
		\end{center}
	    \end{table}
         \centerline {\bf ЗАДАНИЕ НА КУРСОВОЙ ПРОЕКТ}
         \noindent{Студенту \underline{ \textbf{\textit{Хвацковой Антонине Алексеевне}}} группы 6312\\}
         {Тема проекта: \underline{ \textbf{\textit{ «Отладка»}}}}
         	\begin{table}[h]
         	\begin{center}
         		\begin{tabular}{|p{4,5cm}|p{6cm}|p{6cm}|}
         			\hline  
         			Планируемые результаты освоения образовательной программы (компетенции) & Планируемые результаты практики & Содержание задания \\ \hline
         			ОПК-3 ‑ способность применять языки, системы и инструментальные средства программирования в профессиональной деятельности & текст & текст \\ ОК-8 -  способность к самоорганизации и самообразованию & & \\ \hline 
         		\end{tabular}
         	\end{center}
         \end{table}
      \noindent{\\Дата выдачи задания \underline{7 февраля 2022 г.} \\}
      {Срок представления на кафедру отчета о практике \underline{xx мая 2022 г.} \\}
      Руководитель курсового проекта\\
      ассистент каф. ГИиИБ \ \ \ \ \ \ \ \ \ \ \ \ \ \ \ \ \ \ \ \ \ \ \ \ \ \ \ \ \ \ \ \ $\underset{\text{\itshape ({\small подпись})}}{\underline{\hspace{0.2\textwidth}}}$ \ \ \ \  Борисов А.Н.\\
      Задание принял к исполнению\\
      студент группы \textnumero  \ \ \ \ \ \ \ \ \ \ \ \ \ \ \ \ \ \ \ \ \ \ \ \ \ \ \ \ \ \ \ \ \ \ \ \ \ \ \ \ $\underset{\text{\itshape ({\small подпись})}}{\underline{\hspace{0.2\textwidth}}}$ \ \ \ \  Хвацкова А.А.\\
\end{titlepage}


{\bf РЕФЕРАТ}

\vskip12pt
{\bf Пояснительная записка к курсовому проекту: }

ОТЛАДЧИК

Цель работы –  написание отладчика в операционной системе Linux с помощью языка программирования С++ с реализацией таких функций, как запуск, остановка и продолжение выполнения программы, установка шага, распечатка трассировки, выведение значений переменных и другие.

\tableofcontents

\chapter*{Введение}
\chapter{Используемые алгоритмы}
\section{Компиляция и отладка}
Добрый день
\section{Начало работы}
Здравствуйте

\end{document}
